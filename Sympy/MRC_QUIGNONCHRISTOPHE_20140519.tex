

%% This file was auto-generated by IPython.
%% Conversion from the original notebook file:
%%
\documentclass[11pt,english]{article}

%% This is the automatic preamble used by IPython.  Note that it does *not*
%% include a documentclass declaration, that is added at runtime to the overall
%% document.

\usepackage{amsmath}
\usepackage{amssymb}
\usepackage{graphicx}
\usepackage{grffile}
\usepackage{ucs}
\usepackage[utf8x]{inputenc}

% Scale down larger images
\usepackage[export]{adjustbox}

%fancy verbatim
\usepackage{fancyvrb}
% needed for markdown enumerations to work
\usepackage{enumerate}

% Slightly bigger margins than the latex defaults
\usepackage{geometry}
\geometry{verbose,tmargin=3cm,bmargin=3cm,lmargin=2.5cm,rmargin=2.5cm}

% Define a few colors for use in code, links and cell shading
\usepackage{color}
\definecolor{orange}{cmyk}{0,0.4,0.8,0.2}
\definecolor{darkorange}{rgb}{.71,0.21,0.01}
\definecolor{darkgreen}{rgb}{.12,.54,.11}
\definecolor{myteal}{rgb}{.26, .44, .56}
\definecolor{gray}{gray}{0.45}
\definecolor{lightgray}{gray}{.95}
\definecolor{mediumgray}{gray}{.8}
\definecolor{inputbackground}{rgb}{.95, .95, .85}
\definecolor{outputbackground}{rgb}{.95, .95, .95}
\definecolor{traceback}{rgb}{1, .95, .95}

% new ansi colors
\definecolor{brown}{rgb}{0.54,0.27,0.07}
\definecolor{purple}{rgb}{0.5,0.0,0.5}
\definecolor{darkgray}{gray}{0.25}
\definecolor{lightred}{rgb}{1.0,0.39,0.28}
\definecolor{lightgreen}{rgb}{0.48,0.99,0.0}
\definecolor{lightblue}{rgb}{0.53,0.81,0.92}
\definecolor{lightpurple}{rgb}{0.87,0.63,0.87}
\definecolor{lightcyan}{rgb}{0.5,1.0,0.83}

% Framed environments for code cells (inputs, outputs, errors, ...).  The
% various uses of \unskip (or not) at the end were fine-tuned by hand, so don't
% randomly change them unless you're sure of the effect it will have.
\usepackage{framed}

% remove extraneous vertical space in boxes
\setlength\fboxsep{0pt}

% codecell is the whole input+output set of blocks that a Code cell can
% generate.

% TODO: unfortunately, it seems that using a framed codecell environment breaks
% the ability of the frames inside of it to be broken across pages.  This
% causes at least the problem of having lots of empty space at the bottom of
% pages as new frames are moved to the next page, and if a single frame is too
% long to fit on a page, will completely stop latex from compiling the
% document.  So unless we figure out a solution to this, we'll have to instead
% leave the codecell env. as empty.  I'm keeping the original codecell
% definition here (a thin vertical bar) for reference, in case we find a
% solution to the page break issue.

%% \newenvironment{codecell}{%
%%     \def\FrameCommand{\color{mediumgray} \vrule width 1pt \hspace{5pt}}%
%%    \MakeFramed{\vspace{-0.5em}}}
%%  {\unskip\endMakeFramed}

% For now, make this a no-op...
\newenvironment{codecell}{}

 \newenvironment{codeinput}{%
   \def\FrameCommand{\colorbox{inputbackground}}%
   \MakeFramed{\advance\hsize-\width \FrameRestore}}
 {\unskip\endMakeFramed}

\newenvironment{codeoutput}{%
   \def\FrameCommand{\colorbox{outputbackground}}%
   \vspace{-1.4em}
   \MakeFramed{\advance\hsize-\width \FrameRestore}}
 {\unskip\medskip\endMakeFramed}

\newenvironment{traceback}{%
   \def\FrameCommand{\colorbox{traceback}}%
   \MakeFramed{\advance\hsize-\width \FrameRestore}}
 {\endMakeFramed}

% Use and configure listings package for nicely formatted code
\usepackage{listingsutf8}
\lstset{
  language=python,
  inputencoding=utf8x,
  extendedchars=\true,
  aboveskip=\smallskipamount,
  belowskip=\smallskipamount,
  xleftmargin=2mm,
  breaklines=true,
  basicstyle=\small \ttfamily,
  showstringspaces=false,
  keywordstyle=\color{blue}\bfseries,
  commentstyle=\color{myteal},
  stringstyle=\color{darkgreen},
  identifierstyle=\color{darkorange},
  columns=fullflexible,  % tighter character kerning, like verb
}

% The hyperref package gives us a pdf with properly built
% internal navigation ('pdf bookmarks' for the table of contents,
% internal cross-reference links, web links for URLs, etc.)
\usepackage{hyperref}
\hypersetup{
  breaklinks=true,  % so long urls are correctly broken across lines
  colorlinks=true,
  urlcolor=blue,
  linkcolor=darkorange,
  citecolor=darkgreen,
  }

% hardcode size of all verbatim environments to be a bit smaller
\makeatletter 
\g@addto@macro\@verbatim\small\topsep=0.5em\partopsep=0pt
\makeatother 

% Prevent overflowing lines due to urls and other hard-to-break entities.
\sloppy




\begin{document}




\section{Mathematics for Robotics and Control - Assignment 7: Ordinary
Differential equations}


For this assignment, you will be working with symbolic mathematics using
the SymPy package for Python. In order to be able to work with the
package efficiently, please familiarize yourself with SymPy by working
through the tutorial:

\href{http://docs.sympy.org/dev/tutorial/tutorial.en.html\#tutorial}{SymPy
Tutorial}

Also, make sure you know how SymPy differes from other CAS like Maple or
Mathematica, which you may be familiar with. This information is
detailed in the following document:

\href{http://docs.sympy.org/dev/gotchas.html}{SymPy Gotchas}

There is also an FAQ:

\href{https://github.com/sympy/sympy/wiki/Faq}{SymPy FAQ}

Also, be sure to read the documentation/help on SymPy's ODe module.
Simply evaluate the code in the next cell to obtain said documentation.
Note that using ? allows you to obtain help on any function, module etc.
when used in IPython.

\begin{codecell}


\begin{codeinput}
\begin{lstlisting}
from sympy import *
#import sympy
#?sympy.ode
\end{lstlisting}
\end{codeinput}

\end{codecell}

Let us now start the assignment by examining some differential
equations. First, have a look at the following quick example below to
see how to enter code into the IPython notebook and how to work with
SymPy.

\begin{codecell}


\begin{codeinput}
\begin{lstlisting}
# enable pretty printing
#sympy.
init_printing(use_latex=True)
init_printing(use_unicode=True)
#Sorry, no time for latex...
\end{lstlisting}
\end{codeinput}

\end{codecell}

\begin{codecell}


\begin{codeinput}
\begin{lstlisting}
# Declare a symbolic variable x. Note that the x on the left side of the assignment is the Python variable name,
# while the "x" on the right hand side denotes the symbol name. It makes a lot of sense to keep those names the same.
x = Symbol("x")
# y is a function of x, i.e. y(x)
y = Function("y")(x)
# We use y_ to denote y'(x), i.e. the first derivative of y w.r.t. x
y_ = Derivative(y, x)
# ...an alternative way of writing this would be:
y_ = y.diff(x)

# Now let's examine a differential equation
#
#                x   d       
# y(x)\cdotsin(x) + e  = --(y(x))
#                    dx      
# 
eq1 = y * sin(x) + exp(x) - y_

# We now determine a solution for this differential equation
pprint(dsolve(eq1))
\end{lstlisting}
\end{codeinput}
\begin{codeoutput}


\begin{Verbatim}[commandchars=\\\{\}]
/                                   
| /               x\  cos(x)        
| \ y(x)*sin(x) + e /*e       dx = C1
/
\end{Verbatim}

\end{codeoutput}

\end{codecell}

For your first assignment, solve the following three ODes \emph{by hand}
and verify your results via SymPy. Please include the steps of your
solutions in the input cell below using TeX syntax. You can find an
introduction to mathematical expressions in LaTeX
\href{http://en.wikibooks.org/wiki/LaTeX/Mathematics}{here}. In order
for the IPython notebook to evaluate your mathematical expressions,
enclose them in \$, i.e.

\begin{verbatim}
$\sum_{i=1}^{10} t_i$ 
\end{verbatim}
will display the following expression: $\sum_{i=1}^{10} t_i$

\begin{center}\rule{3in}{0.4pt}\end{center}

\paragraph{Assignment 7.1: Solve the following ordinary differential
equations by hand and verify your results using SymPy.}


equation 1.1: $y^\prime = 5 \cdot y$

$\ln (y) = 5x+c$

$e^{5x}$

\begin{codecell}


\begin{codeinput}
\begin{lstlisting}
# Insert code to verify your solution for equation 1.1 here and evaluate it
x = Symbol('x')
y = Function('y')(x)
y_dash = Derivative(y, x)
ode1 = eq(y_dash, 5*y)
pprint(ode1)

x = Symbol('x')
y = Function('y')(x)
c = Symbol('c')
#My solution:
integration = exp(5*x)
s = eq(y, integration)
pprint(s)

checkodesol(ode1, s)
\end{lstlisting}
\end{codeinput}
\begin{codeoutput}


\begin{Verbatim}[commandchars=\\\{\}]
d                
--(y(x)) = 5*y(x)
dx               
        5*x
y(x) = e
\end{Verbatim}



\begin{verbatim}
(True, 0)
\end{verbatim}



\end{codeoutput}

\end{codecell}

equation 1.2: $\frac{\mathrm d y}{\mathrm d x} = -2 \cdot x \cdot y$

$\ln(y) = -x^{2+c}$

$c * e^{-x^{2}}$

\begin{codecell}


\begin{codeinput}
\begin{lstlisting}
# Insert code to verify your solution for equation 1.2 here and evaluate it
x = Symbol('x')
y = Function('y')(x)
dydx = Derivative(y, x)

ode2 = eq(dydx, -2*x*y)

c = Symbol('c')
#My solution:
integration = c*exp(-(x**2))
s = eq(y, integration)
pprint(s)

checkodesol(ode2, s)
\end{lstlisting}
\end{codeinput}
\begin{codeoutput}


\begin{Verbatim}[commandchars=\\\{\}]
2
          -x 
y(x) = c*e
\end{Verbatim}



\begin{verbatim}
(True, 0)
\end{verbatim}



\end{codeoutput}

\end{codecell}

equation 1.3: $\frac{\mathrm d y}{\mathrm d t} = y^2$

$1/y = t * c$

$y = -1/(c+t)$

\begin{codecell}


\begin{codeinput}
\begin{lstlisting}
# Insert code to verify your solution for equation 1.3 here and evaluate it
t = Symbol('t')
y = Function('y')(t)
dydx = Derivative(y, t)
ode3 = eq(dydx, y**2)

c = Symbol('c')
#My solution:
s = eq(y, -1/(c+t))
pprint(s)

checkodesol(ode3, s)
\end{lstlisting}
\end{codeinput}
\begin{codeoutput}


\begin{Verbatim}[commandchars=\\\{\}]
-1  
y(t) = -----
       c + t
\end{Verbatim}



\begin{verbatim}
(True, 0)
\end{verbatim}



\end{codeoutput}

\end{codecell}

\emph{Assignment 7.1 took me} \emph{minutes.}

\begin{center}\rule{3in}{0.4pt}\end{center}

\paragraph{Assignment 7.2: Determine if the following equations are
linear and insert your solutions below.}


\emph{Note: simply stating that an equation is linear or not is not
sufficient, please provide arguments/proof as to why this is the case.}

\begin{enumerate}[1.]
\item
  $y^\prime = sin(x) \cdot y + e^x$
\item
  $y^\prime + x \cdot y = e^x \cdot y$
\item
  $y^\prime + \frac{x}{y} = 0$
\item
  $x \cdot y^\prime + y = \sqrt{y}$
\end{enumerate}

\begin{codecell}


\begin{codeinput}
\begin{lstlisting}
#1. is a linear function becaus its highest power is 1
#2. is a linear function becaus its highest power is 1
#3. is not a linear function because it is not a function. (Circle)
#4. is not linear because it has no solution at x = 0
\end{lstlisting}
\end{codeinput}

\end{codecell}

\emph{Assignment 7.2 took me} \emph{minutes.}

\begin{center}\rule{3in}{0.4pt}\end{center}

\paragraph{Assignment 7.3: Solve the following ordinary differential
equations and specify an integrating factor. Insert your solutions
below.}


\href{http://en.wikipedia.org/wiki/Integrating\_factor}{Wikipedia:
Integrating Factor}

\href{http://www.cse.salford.ac.uk/profiles/gsmcdonald/H-Tutorials/ordinary-differential-equations-integrating-factor.pdf}{Tutorial
on how to use the integrating factor method to solve ODes} (PDF)

\begin{enumerate}[1.]
\item
  $y^\prime -3 \cdot y = 6$
\item
  $y^\prime + \frac{4}{x} \cdot y = x^4$
\end{enumerate}

\begin{codecell}


\begin{codeinput}
\begin{lstlisting}
#1.

x = Symbol('x')
y = Function('y')(x)
dydx = Derivative(y, x)

P = Function('P')(x)
Q = Function('Q')(x)

# y' -3y = 6
#  -> P(x) = -3y
#     Q(x) = 6

P = -3
Q = 6

# 
# IF: e^(-3x)

IF = exp(integrate(P, x))

# 
# Multiply with IF
#     (y' -3y) * e^(-3x) = 6e^(-3x)
# e^(-3x)  - 3e^()-x3)*y = 6e^(-3x)
#       y'(e(-3x)*y -3 ) = 6e^(-3x)

left  = IF * (dydx + P * y )
right = Q * IF

# 
# Integrate:
#         e^(-3x) - 3x = 6e^(-3x)
# 6*((e^(-3x))/(-3))*C = -2e^(-3x) + C
#             e^(-3x)y = 3-2e^(-3x)-+C
#                    y = (3-C)*-2e^(-3x)
#

left  = integrate(left)
right = integrate(right)

s = eq(left, right)
pprint(s)
\end{lstlisting}
\end{codeinput}
\begin{codeoutput}


\begin{Verbatim}[commandchars=\\\{\}]
-3*x       -3*x
y(x)*e     = -2*e
\end{Verbatim}

\end{codeoutput}

\end{codecell}

\begin{codecell}


\begin{codeinput}
\begin{lstlisting}
# 2.
x = Symbol('x')
y = Function('y')(x)
dydx = Derivative(y, x)

P = Function('P')(x)
Q = Function('Q')(x)

# y' + 4/x * y = x^4
#  -> P(x) = 4/x
#  -> Q(x) = x^4

P = 4/ x
Q = x ** 4

# IF: x^4

IF = exp(integrate(P, x))

# 
# Multiply with IF:
#    x^4 * (y' +4/x * y) = x^4 * x^4
#       x^3 * (y' +4y/x) = x^4 * x^4
#          x^3(y*x + 4y) = x^8
#            y'(xy + 4y) = x^5

left  = IF * (dydx + P * y )
right = Q * IF

# 
# Integrate:
# 
# y'(xy + 4y) = x^5
#     x^4 * y = 1/9 x^9 + C
#           y = x^5  / 9 + C

left  = integrate(left)
right = integrate(right)

s = eq(left, right)
pprint(s)
\end{lstlisting}
\end{codeinput}
\begin{codeoutput}


\begin{Verbatim}[commandchars=\\\{\}]
9
 4        x 
x *y(x) = --
          9
\end{Verbatim}

\end{codeoutput}

\end{codecell}

\emph{Assignment 7.3 took me} \emph{minutes.}

\begin{center}\rule{3in}{0.4pt}\end{center}

\paragraph{Assignment 7.4: Solve the following differental equations}


Remember that a second order differential equation
$y^{\prime \prime} + a_1 \cdot y^\prime + a_0 \cdot y = 0$ with
constants $a_{0, 1}$ corresponds to the characteristic equation
$\lambda^2 + a_1 \cdot \lambda + a_0 = 0$, which can be factored into
$(\lambda - \lambda_1) \cdot (\lambda - \lambda_2) = 0$. This is useful
since the general solution of
$y^{\prime \prime} + a_1 \cdot y^\prime + a_0 \cdot y = 0$ can be
obtained directly by determining the roots $\lambda_1$ and $\lambda_2$
if the coefficients are constant and the differential equation is
linear. Determine the roots of the following differential equations and
specify their solutions. See the Wikipedia article on the
\href{http://en.wikipedia.org/wiki/Characteristic\_equation\_(calculus)}{characteristic
equation} for the three distinctive cases that determine how to obtain
the solution based on what the roots are.

Insert your solutions below and include code to proof your solutions are
correct. Hint: use the sympy.ode.checkodesol function to validate your
results.

\begin{enumerate}[1.]
\item
  $y^{\prime \prime} - y^\prime - 2 \cdot y = 0$
\item
  $y^{\prime \prime} - 3 \cdot y^\prime + 4 \cdot y = 0$
\item
  $y^{\prime \prime} + 4 \cdot y = 0$
\end{enumerate}

\begin{codecell}


\begin{codeinput}
\begin{lstlisting}
#1.
#r^2 -1r -2 = 0
#(r +1)(r -2) = 0

#->
#r_1 = -1
#r_2 = +2

#    y(t) =   C_1 e^(-1t) +  C_2e^(2t)
#y^(t)(t) = -1C_1 e^(-1t) + 2C_2

#0 = y(0)     =   C_1 +  C_2
#1 = y^(t)(0) = -1C_1 + 2C_2

#C_1 = 1
#C_2 = -1

#->
#y(t)= 1e^(-1t) -1e^(2t)

t = Symbol('t')
y = Function('y')(t)

d = Derivative(Derivative(y)) -Derivative(y)- 2*y         
s = 1 *exp(-1*t) -1*exp(2*t)

pprint(dsolve(eq(s, d)))
#looks similar to me
\end{lstlisting}
\end{codeinput}
\begin{codeoutput}


\begin{Verbatim}[commandchars=\\\{\}]
/     t\  -t   /     t\  2*t
y(t) = |C1 - -|*e   + |C2 - -|*e   
       \     3/       \     3/
\end{Verbatim}

\end{codeoutput}

\end{codecell}

\begin{codecell}


\begin{codeinput}
\begin{lstlisting}
#2.
#y^2 - 3y +4
#(--3(+-)sqrt(-3^2 -4 *1*4))/(2*1) #quadratic equation

#sqrt(5/4)

#r1 = 3(+-)sqrt(9-16)
#r2 = 11.5 (+-)sqrt(-5/4)i

#Complex roots:
#a = 1.5
#b = 1.118

#y=(c1 +c2)exp(1,5*t) cos(1.118t) + i(c1 -c2)exp(1.5t) sin(1.118t)

t = Symbol('t')
y = Function('y')(t)

b = sqrt(5/4)

c1, c2 = symbols('c1 c2')

d = Derivative(Derivative(y)) -Derivative(3*y) + 4*y       
s = (c1 + c2) * exp(1.5*t) * cos(b*t) + I*(c1 -c2)*exp(1.5*t) *sin(b*t)
pprint(dsolve(eq(s, d)))
#looks similar to me...
\end{lstlisting}
\end{codeinput}
\begin{codeoutput}


\begin{Verbatim}[commandchars=\\\{\}]
/                                                                      
       |                                                                      
       |      /  ___  \         /  ___  \   4*(i*c1*sin(t) + c1*cos(t) - i*c2*
       |      |\/ 7 *t|         |\/ 7 *t|                                     
y(t) = |C1*sin|-------| + C2*cos|-------| + ----------------------------------
       \      \   2   /         \   2   /                                    3

                        /  ___  \                                             
                       2|\/ 7 *t|                                             
sin(t) + c2*cos(t))*sin |-------|   4*(i*c1*sin(t) + c1*cos(t) - i*c2*sin(t) +
                        \   2   /                                             
--------------------------------- + ------------------------------------------
                                                                     3        

                /  ___  \\        
               2|\/ 7 *t||        
 c2*cos(t))*cos |-------||     3/2
                \   2   /| / t\   
-------------------------|*\ e /   
                         /
\end{Verbatim}

\end{codeoutput}

\end{codecell}

\begin{codecell}


\begin{codeinput}
\begin{lstlisting}
#3.
#y^2  = -4

#Complex roots:
#a = 0
#b = 2

#y=(c1 +c2) cos(2t) + i(c1 -c2) sin(2t)

##Proof

t = Symbol('t')
y = Function('y')(t)

c1, c2 = symbols('c1 c2')

d = Derivative(Derivative(y)) + 4*y       
s = (c1 + c2)* cos(2*t) + I*(c1 -c2) *sin(2*t)

pprint(dsolve(eq(s, d)))
#looks similar to me...
\end{lstlisting}
\end{codeinput}
\begin{codeoutput}


\begin{Verbatim}[commandchars=\\\{\}]
2                                      
                     i*t*(-c1 + c2)*sin (2*t)*cos(2*t)   (-2*i*c1*t + c1 + 2*i
y(t) = C2*cos(2*t) + --------------------------------- + ---------------------
                                     4                                       8

               3                                                              
*c2*t + c2)*cos (2*t)   /       /c1   c2\   /c1   c2\            /  i*c1   i*c
--------------------- + |C1 + t*|-- + --| + |-- + --|*sin(4*t) + |- ---- + ---
                        \       \ 4    4 /   \ 16   16/            \   16     16

                                               
2\            /i*c1   i*c2\    2     \         
-|*cos(4*t) + |---- - ----|*cos (2*t)|*sin(2*t)
 /            \ 8      8  /          /
\end{Verbatim}

\end{codeoutput}

\end{codecell}

\emph{Assignment 7.4 took me} \emph{minutes.}

\begin{center}\rule{3in}{0.4pt}\end{center}


\emph{Use this button to create a .txt file containing the time in
minutes you spent working on the assignments. Make sure to include your
name in the textbox below. The file will be created in the current
directory.}

Student's name:



\end{document}

